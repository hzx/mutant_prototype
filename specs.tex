\documentclass[a4paper,12pt]{book}
%\usepackage{microtype}
%\usepackage{parskip}
\usepackage[english]{babel}
\usepackage{blindtext}
\begin{document}
\title{Mutant language specification 0.1}
\maketitle

\chapter{Overview}

Mutant - domain specific language for rapid web development. Eliminate
90\% development effort. Make development workflow faster, working in two
modes - debug, release.

Debug mode create fast build, continuous code generation to destination
languages, build on the fly.

Release mode create optimized build, minify css, js, html, inline images into
css.

\section{Modules, namespaces}
Code consists of modules, each module have same namespace name.

\section{Simple types}

\begin{verbatim}
void, int, float, bool, string, datetime, tag.
\end{verbatim}

\section{Function type}

Declare alias and function signature;

\begin{verbatim}
typedef FuncType {type} ({params});
\end{verbatim}

Anonymous function declaration

\begin{verbatim}
typedef EventHandler void (Event e);
var clickHandler = void (Event e) { };
\end{verbatim}

\section{Collections}

\section{Array declaration}

\begin{verbatim}
type ident[length] = [a1, a2, an];
\end{verbatim}

\section{Dictionary declaration}

\begin{verbatim}
type ident{key_type:value_type} = {k1:v1, k2:v2, kn:vn};
\end{verbatim}

\section{Type definition, alias}

\begin{verbatim}
typedef alias declaration;
\end{verbatim}

\section{User types}

enum, struct, class, interface\\

\begin{verbatim}
enum ident {
  ident = {litint};
}

interface ident {
  {type} ident({params});
}

class ident extends ident_base implements iface1, iface2, ifacen {
  () { {function_body} }
  {type} ident = {expr};
  {type} ident({params}) { {function_body} }
}

struct ident extends ident_base {
  {type} ident { {struct_var_params} };
}
\end{verbatim}

Inheritance:\\

interface, struct, classes may be extended. Function overloading accepted only
with overload keyword. For interface inheritance not allowed, for class
allowed from only one class and implement many interfaces (methods in
interfaces must have distinct names). For struct allowed only from one
superstruct.

\begin{verbatim}
\end{verbatim}

Function body:\\

\begin{verbatim}
{var_declaration} | {return_operator} | {if_operator} |
{for_operator} | {foreach_operator} | {while_operator}
\end{verbatim}

Variable body:\\

\begin{verbatim}
{bool_literal} | {int_literal} | {float_literal} | {string_literal} |
{datetime_literal} |
{array_body} | {dictionary_body} | 
{anonymous_function} | {expr}
\end{verbatim}

Class instantiation:\\

\begin{verbatim}
var app = App();
\end{verbatim}

Take members and methods:\\

Cascade operator:

\begin{verbatim}
App.instance()
  ..title = "hello"
  ..setStatistic("user-01")
  ..init();
\end{verbatim}
\end{document}
